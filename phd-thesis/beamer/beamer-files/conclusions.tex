\section{Trabajo futuro y conclusiones  }

\subsection{RAS con campo externo}

\begin{frame}[t]{RAS con campo externo }
\begin{tikzpicture}[remember picture, overlay]
\node<1->[anchor=north west,xshift=0cm,yshift=-1cm,scale=1.5,blue](t1) at (current page.north west){Muestra sim\'etrica $d_{1}=d_{2}$};
\node<1-3>[anchor= north east,xshift=-2cm,yshift=-0.2 5cm] at (current page.north east){\includegraphics[scale=0.25]{Figures/STRUCTURES/FIG1}};
\foreach [count=\xi from 2]\x in {1,...,2}{
	\node<\xi>[anchor=north west,xshift=-0.15cm](sp1) at (t1.south west){\includegraphics[page={\x},width=0.6\textwidth]{Figures/SPOT/SPOT-2}};
}

\foreach [count=\xi from 4]\x in {1,...,6}{
	\node<\xi>[anchor=north west,xshift=-0.15cm,yshift=0.25cm](sp2) at (t1.south west){\includegraphics[page={\x},width=0.8\textwidth]{Figures/SPOT/SPOT-1}};
}

\node<10->[anchor=north east,xshift=-21mm,yshift=-7mm](sp3) at (current page.north east){\includegraphics[page={1},width=0.4\textwidth]{Figures/SPOT/SPOT-3}};
\node<10->[anchor=east,xshift=0.8cm,yshift=-10mm](AES) at (sp3.south west){
	\animategraphics[autoplay,loop,width=0.65\textwidth]{2}{FIGURES/AEF-RAS/AEF-RAS}{0}{5}};



\end{tikzpicture}
\end{frame}


\begin{frame}[t]{Conclusiones}
\vspace{-2cm}
\begin{tikzpicture}[remember picture, overlay]
\node[anchor=center, xshift = -1cm,yshift = 1cm] at (current page.center) {
	\begin{tcolorbox}[enhanced,
					 title=Conclusiones y perspectivas,
				     halign=center,
					 width=9.5cm,
					 left=1mm,
					 top=1mm,
					 bottom=1mm,
					 right=1mm,
					 boxsep=0mm,
					 fonttitle=\bfseries,
					coltitle=blue!25!black,
					attach boxed title to top center={yshift=-2mm,
					yshifttext=-1mm},
					boxed title style={colframe=blue!75!black,
										colback=myblue}]
					\hspace{-1cm}
					\begin{itemize}	
					\justifying
							\item<1->El articulo esta en etapa final para ser enviado.
                            \item<2-> Experimentos de RAS con campo aplicado
							\item<3->  Se realizar\'an experimentos de \emph{Fotorreflectancia en modo exitaci\'on} para entender las transiciones de los Triones y correlacionar los resultados con los experimentos de campo aplicado.
					\end{itemize}          
	\end{tcolorbox}
};

\end{tikzpicture}



\end{frame}