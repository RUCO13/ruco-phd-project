\section{Introducción}

\subsection{Semiconductores}
\begin{frame}[t]
\frametitle{\secname}
\framesubtitle{\subsecname}
\vspace{-0.5cm}
\begin{tikzpicture}[remember picture, overlay]
\node<1->[anchor = north west, text width =0.7\textwidth, yshift = -0.75cm](txt) at (current page.north west) {
	\begin{tcolorbox}[title =\subsecname, width=\textwidth]
	\begin{itemize}
	\item<1-> Aplicaciones potenciales de los pozos cuanticos.\footnotemark[1] 
	\end{itemize}
	\end{tcolorbox}	
};


\node<1->[anchor=south west,xshift=0mm,yshift=5mm] at (current page.south west){\includegraphics[width=\textwidth]{./beamer-figures/introduction/app1-v2.png}};
\end{tikzpicture}

\footnotetext[1]{Li, J. et al. (2020). Novel Nitride LED Technology. In: III-Nitrides Light Emitting Diodes: Technology and Applications. Springer Series in Materials Science, vol 306. Springer, Singapore.}
\end{frame}



\subsection{Estructura de Bandas}
\begin{frame}[t]
\frametitle{Introducci\'on}
\framesubtitle{Estructura de Bandas}
\vspace{-0.5cm}
\begin{tikzpicture}[remember picture, overlay]
\node<1->[anchor = north west, text width =0.7\textwidth, yshift = -0.75cm](txt) at (current page.north west) {
	\begin{tcolorbox}[title =\subsecname, width=\textwidth]
	\begin{itemize}
	\item<1-> Dicta el comportamiento de los electrones en un solido
	\item<2-> En un solido $\approx 10^{23}$ atomos\\$\rightarrow$\text{\color{red}problema complejo de muchos cuerpos}
	\item<3-> Hamiltoniano de un solido
	\item<4-> Gracias al Teorema de Bloch\\ $\rightarrow$ potencial periodico
	\item<5-> Ecuacion de Schr\"odinger en terminos de un el\'ectron. 
	\end{itemize}
	\end{tcolorbox}	
};


\node<1->[anchor=north east,xshift=-2cm,yshift=-1cm] at (current page.north east){\includegraphics[width=0.35\textwidth]{../../../scripts/structures/GaAs-2}};
\node<1>[anchor=south west,yshift=0mm,inner sep=0mm](s) at (current page.south west){\includegraphics[width=\textwidth]{../../figures/chapter-1/solid-sort/build/solid-sort}};



\node<3>[anchor=center,text width=\textwidth,font=\sffamily,xshift=-1cm,yshift=-2cm,scale=1.2] at (current page.center){
\begin{equation*}
	\begin{split}
	H  =  &\dfrac{1}{2M}\sum\limits_{i=1}^{N_{n}} \bff{P}_{j}^{2} + \dfrac{1}{2m_{0}} \sum\limits_{j=1}^{N_{e}} \bff{p}_{j}^{2} + \dfrac{Z_{e}^{2}}{2} \sum\limits_{i,j=1,i\neq j}^{N_{n}} V_{c}\left(\bff{R}_{i}-\bff{R}_{j}\right)-Z_{e}\sum\limits_{i=1}^{N_{n}}\sum\limits_{j=1}^{N_{e}}V_{c}\left(\bff{r}_{j}-\bff{R}_{i}\right) \\
	& + \dfrac{1}{2} \sum\limits_{i,j=1,i\neq j}^{N_{e}} V_{c} \left(\bff{r}_{i}-\bff{r}_{j}\right)
	\end{split}
\end{equation*}
};

\node<4->[anchor=south west,opacity=0.5](i1) at (current page.south west){\includegraphics[width=\textwidth,trim = {0cm 0cm 0cm 2cm},clip]{beamer-figures/introduction/atom-line.png}};
\node<4->[anchor=center,yshift=0cm,inner sep=0mm,yshift=0cm](bloch) at (i1.center){\animategraphics[autoplay,loop,width=\textwidth]{8}{beamer-figures/introduction/b1}{}{}};

\node<4->[anchor=south west,text width=\textwidth,scale=0.7] at (current page.south west) {Image credit: from \href{https://commons.wikimedia.org/wiki/File:Standing_wave.gif}{Wikimedia Commons}, public domain};
\node<5->[anchor=south west,yshift=-0.5cm,inner sep=0mm](schrodinger)at(bloch.north){$\left[-\dfrac{\hbar^2}{2m_{0}}\nabla^2 + U(\rv)\right]\psi (r)=E\psi(\rv)$}; 


\end{tikzpicture}
\end{frame}


\begin{frame}[t]
	\frametitle{Introducci\'on}
	\framesubtitle{Bandas de Valencia y Conducci\'on para un semiconductor en Bulto}
	\vspace{-0.5cm}
	\begin{tikzpicture}[remember picture, overlay]


%==========> parte 2

\node<1->[anchor=north east,xshift=-2cm,yshift=-0.9cm] at (current page.north east){\includegraphics[width=0.3\textwidth]{../../../scripts/structures/GaAs-2}};
% \node<3->[anchor=south east,text width=0.2\textwidth,font=\sffamily,xshift=-5mm,yshift=-8mm,scale=1.5,blue,inner sep=0mm](e1) at (s.south east){
% \begin{equation*}
% 	\left(\tikzmarknode{a}{\mathcal{H}}-\varepsilon(\mathbf{k}\lambda)\ket{\mathbf{k\lambda}}\right)=0
% \end{equation*}
% };
\node<1->[anchor=south west,inner sep=0mm,yshift=0mm](bs) at (current page.south west){\includegraphics[width=1.05\textwidth]{../../figures/chapter-1/bands/build/bands01}};
\node<1->[anchor=south west,text width=0.6\textwidth,scale=1.1] at (bs.north west){Estructura de bandas para GaAs en bulto};


\node<1->[anchor=north west,inner sep=0mm,yshift=-8mm](b02) at (current page.north west){\includegraphics[width=0.28\textwidth]{../../figures/chapter-1/bands/build/bands02}};

\node<1->[anchor=north west,inner sep=0mm,xshift=7mm,yshift=-2mm](b03) at (b02.north east){\includegraphics[width=0.33\textwidth]{../../figures/chapter-1/exciton-1/build/x-1}};
\end{tikzpicture}
\end{frame}
	
	
\subsection{Estructuras de baja dimensi\'on}
\begin{frame}[t]
	\frametitle{Introducci\'on}
	\framesubtitle{Estructuras de baja dimensi\'on}
	\vspace{-0.5cm}
	\begin{tikzpicture}[remember picture, overlay]
	\node<1->[anchor = north west, text width =0.75\textwidth, yshift = -0.75cm](txt) at (current page.north west) {
		\begin{tcolorbox}[title =\subsecname, width=\textwidth]
		\begin{itemize}
		\item<1-> Semiconductor en bulto y su estructura de bandas
		\item<2-> Heteroestructuras semiconductoras 
		\item<3-> Heteroestructuras $\rightarrow$ Systemas nanoestructurados
		\item<4-> Primera approximaci\'on a los Pozos Cu\'anticos y su {\textcolor{magenta}{confinamiento}}
		\end{itemize}
		\end{tcolorbox}	
	};


	\node<1->[anchor=north east,xshift=-2cm,yshift=-1cm] at (current page.north east){\includegraphics[width=0.25\textwidth]{../../../scripts/structures/GaAs-2}};
	\node<2>[anchor=south west,xshift=10mm] at (current page.south west){\includegraphics[width=0.59\textwidth]{../../figures/chapter-1/heterostructures/build-ruco/hs-01}};
	\node<3>[anchor=south west,xshift=10mm] at (current page.south west){\includegraphics[width=0.59\textwidth]{../../figures/chapter-1/heterostructures/build-ruco/lds-00}};
	\node<4>[anchor=south west,xshift=0mm](qw1) at (current page.south west){\includegraphics[width=0.48\textwidth]{../../figures/chapter-1/heterostructures/out/qw1}};
	\node<5>[anchor=south west,xshift=0mm](qw2) at (current page.south west){\includegraphics[width=0.55\textwidth]{../../figures/chapter-1/heterostructures/out/qw2}};

%equations
\node<4>[anchor=north west,xshift=5mm,blue](e1) at (qw1.north east){$-\dfrac{\hbar^{2}}{2m^{*}}\dfrac{\partial^{2}}{\partial {z}^{2}}\psi(z)+V(z)\psi(z)=E\psi(z)$};
\node<5->[anchor=north west,xshift=5mm,blue](e1) at (qw1.north east){$-\dfrac{\hbar^{2}}{2m^{*}}\dfrac{d^{2}}{d {z}^{2}}\psi(z)+V(z)\psi(z)=E\psi(z)$};
\node<5->[anchor=north west,blue](e2) at (e1.south west){$E = E_{n} + \dfrac{\hbar^{2}|\boldsymbol{k}_{x,y}|^{2}}{2m^{*}}$};

\end{tikzpicture}
	\end{frame}
	