\documentclass[a4paper,12pt]{article}% Thomas Soell 20120303
\usepackage[T1]{fontenc}
\usepackage[latin1]{inputenc}
\usepackage{ngerman}
\usepackage{amsmath,amsthm}
\usepackage{amsfonts}
\usepackage[psamsfonts]{amssymb}
\renewcommand{\familydefault}{\sfdefault}
\usepackage{mathpazo}
\usepackage[scaled=.95]{helvet}
\usepackage{courier}
%\linespread{1.05}     % Zeilenabstand um 5% erh�hen
\usepackage{upgreek}
\usepackage{multido}      % From PSTricks
\usepackage[dvips]{graphics}
\usepackage[dvips]{graphicx}
\usepackage[dvips,usenames,dvipsnames]{xcolor} %% Farben sind im Dokument xcolor.pdf definiert
\usepackage{pst-plot}
\usepackage{pst-node}
\usepackage{pst-tree}
\usepackage{pst-eps}
\usepackage[tiling]{pst-fill}
\usepackage{pst-text}
\usepackage{pst-grad}
\usepackage{pst-coil}
\usepackage{pstricks-add}


%---------------- Kontinuierlicher Farbverlauf bei Kurven -------------------------
\makeatletter
\pst@addfams{pst-HSB}
\define@key[psset]{pst-HSB}{HueBegin}{% Between 0 and 1
  \def\PstParametricplotHSB@HueBegin{#1}}
\define@key[psset]{pst-HSB}{HueEnd}{% Between 0 and 1
  \def\PstParametricplotHSB@HueEnd{#1}}
\define@boolkey[psset]{pst-HSB}[Pst@]{HSB}[true]{}
% Default values
\psset[pst-HSB]{HueBegin=0,HueEnd=1,HSB=true}
\psset{dimen=outer}

\def\parametricplotHSB{\pst@object{parametricplotHSB}}
\def\parametricplotHSB@i#1#2#3{{%
  \begin@ClosedObj
  \addto@pscode{%
    /t #1 def
    /dt #2 t sub \psk@plotpoints\space div def
    /t t dt sub def
    /Counter 0 def
    1 setlinejoin
    \psk@plotpoints {
      /t t dt add def
      /Counter Counter 1 add def
      #3
      \pst@number\psyunit mul exch
      \pst@number\psxunit mul exch
      1 Counter eq { moveto currentpoint /OldY ED /OldX ED } % First point
        {\ifPst@HSB % Other points than the first one
          /PointY exch def
          /PointX exch def
          Counter \psk@plotpoints\space div
          \PstParametricplotHSB@HueEnd\space
          \PstParametricplotHSB@HueBegin\space sub mul
          \PstParametricplotHSB@HueBegin\space add
          1 1 sethsbcolor
          OldX OldY PointX PointY lineto lineto
          stroke
          PointX PointY moveto
	  /OldX PointX def /OldY PointY def
        \else lineto \fi } ifelse
     } repeat }% fin du code ps
   \end@ClosedObj}%
  \ignorespaces} % fin de la commande PSTricks
\makeatother

%-------------------------------------------------------------------------------

\pagestyle{empty}
\begin{document}

\newhsbcolor{ColorC}{.5 0.8 0}
\newhsbcolor{ColorD}{.5 0.5 0.5}

\rput(0,0){%
%\psgrid(-3,-3)(10,4)%
\pspolygon[linewidth=3pt,linecolor=gray!40,linearc=0,bordercolor=black,border=1.1pt,%  Brennkammer
fillstyle=gradient,gradmidpoint=0,gradangle=10,gradbegin=orange!80,gradend=white]%    mit F�llung
(-1.8,3.45)(-0.9,2)(-0.9,-1.6)(0.9,-1.6)(0.9,2.4)(-0.2,2.4)(-0.88,3.45)(-1.8,3.45)%
\pspolygon[linewidth=3pt,linecolor=gray!40,linearc=0,bordercolor=black,border=1.1pt]%  Brennkammer
(-1.8,3.45)(-0.9,2)(-0.9,-1.6)(0.9,-1.6)(0.9,2.4)(-0.2,2.4)(-0.88,3.45)(-1.8,3.45)%       nur Rahmen
\pspolygon[linewidth=4pt,linecolor=gray!40,linearc=0,bordercolor=black,border=1.1pt]%  Rahmen des Kraftwerks
(-1.85,-2.65)(8.75,-2.65)(8.75,3.45)(-1.85,3.45)%
\pspolygon[fillstyle=solid,fillcolor=black, opacity=1,linewidth=0.8pt]%   Brenner
(-1.35,-0.92)(-1.25,-0.85)(-1,-0.85)(-1,-0.95)(-0.85,-0.95)(-0.85,-1.15)(-1,-1.15)(-1,-1.45)(-1.25,-1.45)(-1.35,-1.38)%
%
%--------------- Flammen ------------------------------------------------------------------
%-------------------------------------------------------------------------------------------
\pscustom[fillstyle=solid,fillcolor=yellow!50,linestyle=none]{%
    \pscurve(-0.87,-1)(-0.4,-0.75)(-0.55,-0.4)(-0.4,-0.13)%
    \pscurve(-0.45,-0.33)(-0.2,-0.53)%
    \psline(-0.2,-0.53)(-0.2,-0.97)%
    \pscurve(-0.2,-0.97)(-0.6,-1.05)(-0.87,-1.01)%
}%
\pscustom[fillstyle=solid,fillcolor=yellow!50,linestyle=none]{%
    \pscurve(-0.2,-0.53)(-0.17,-0.35)(0.1,-0.1)%
    \pscurve(0.1,-0.1)(-0.01,-0.4)(0,-0.6)%
    \pscurve(0,-0.6)(0.04,-0.52)(0.16,-0.46)%
    \pscurve(0.16,-0.46)(0.1,-0.6)(0.17,-0.8)(0,-0.98)(-0.2,-0.97)%
    \psline(-0.2,-0.97)(-0.2,-0.53)%
}%
\pscustom[fillstyle=solid,fillcolor=yellow!50,linestyle=none]{%
    \pscurve(0.1,-0.7)(0.2,-0.6)(0.2,-0.36)(0.4,-0.1)%
    \pscurve(0.4,-0.1)(0.31,-0.29)(0.33,-0.4)(0.39,-0.6)(0.3,-0.8)(0.1,-0.86)%
    \psline(0.1,-0.86)(0.1,-0.7)%
}%
\psframe[linecolor=black,fillstyle=solid,fillcolor=black, opacity=1,linewidth=0.8pt]%
(-1,-1.15)(-0.81,-0.95)%
%-------------------------------------------------------------------------------------------
%--------------- Kondensator ---------------------------------------------------------------
%-------------------------------------------------------------------------------------------
%\psline[linewidth=1.47mm,linecolor=white,linearc=0.15,bordercolor=black,border=1.1pt](5.25,0.285)(5.25,0.9)%
%\psline[linewidth=1.47mm,linecolor=white,linearc=0.15,bordercolor=black,border=1.1pt](6.4,0.285)(6.4,0.9)%
\pscustom[linestyle=none,fillstyle=gradient,gradmidpoint=0,gradangle=0,gradbegin=white,gradend=magenta!80]{%
\psarcn[liftpen=0](6.8,-0.2){2}{190}{165}%
\psline(! 5.162 2 15 sin mul 0.2 sub)(5.162,0.92)(5.338,0.92)(! 5.338 2 15 sin mul 0.2 sub)(! 6.312 2 15 sin mul 0.2 sub)(6.312,0.92)(6.488,0.92)(! 6.488 2 15 sin mul 0.2 sub)%
\psarcn[liftpen=0](4.8,-0.2){2}{15}{-10}%
\closepath%
}
\pscustom[fillstyle=gradient,linestyle=none,gradmidpoint=0,gradangle=0,gradbegin=blue!70,gradend=cyan!60]{%
\psarcn[liftpen=0](6.8,-0.2){2}{195}{190}%
\psarcn[liftpen=0](4.8,-0.2){2}{-10}{-15}%
\closepath%
}
\pscustom[linewidth=1.0pt]{%
\psarcn[liftpen=0](6.8,-0.2){2}{195}{165}%
\psline(! 5.162 2 15 sin mul 0.2 sub)(5.162,0.92)(5.338,0.92)(! 5.338 2 15 sin mul 0.2 sub)(! 6.312 2 15 sin mul 0.2 sub)(6.312,0.92)(6.488,0.92)(! 6.488 2 15 sin mul 0.2 sub)%
\psarcn[liftpen=0](4.8,-0.2){2}{15}{-15}%
\closepath%
}
%
%-----------------------------------------------------------------------------------------
%----  Wasserwendel----------------------------------------------------------------------
%-----------------------------------------------------------------------------------------
\psset{coilheight=0.495,coilwidth=1.3cm,coilaspect=52}%
\rput{90}(0,0){\psCoil[doubleline=true,linecolor=black,linewidth=0.07cm]{250}{720}}%
\rput(0.235,0.4){%
\parametricplotHSB[plotpoints=500,linewidth=1.73mm,HueBegin=0.6,HueEnd=0.84]{270}{90}{0.88 t cos mul 0.365 t sin mul}}%
%%
%\psparametricplot[plotpoints=600,linecolor=white]{0}{360}{0.88 t cos mul 0.365 t sin mul }
%\pscurvepoints[plotpoints=600]{0}{360}{0.88 t cos mul 0.365 t sin mul }{P}%
%\pspolylineticks[Os=0,Ds=.2,ticksize=0 0]{P}{ ds }{0}{90}%
%\definecolorseries{ctest}{hsb}{last}{magenta!80}{cyan!80!blue!30}
%\resetcolorseries[88]{ctest}%
%\multido{\iA=0+1,\iB=1+1}{87}{\psline[linewidth=2pt,linecolor=ctest!![\iB](PTick\iA)(PTick\iB)}%
%%
%\parametricplotHSB[plotpoints=500,linewidth=1.73mm,HueBegin=0.6,HueEnd=0.84]{270}{90}{-3 t 4 mul cos mul 3 t 4 mul sin mul}}%
\rput{90}(0,0){\psCoil[doubleline=true,linecolor=black,linewidth=0.07cm]{600}{1200}}
\rput{90}(0,0){\psCoil[doubleline=true,linecolor=magenta,linewidth=0.045cm]{470}{1200}}
\rput{90}(0,0){\psCoil[doubleline=true,linecolor=black,linewidth=0.07cm]{850}{1400}}
\rput{90}(0,0){\psCoil[doubleline=true,linecolor=magenta,linewidth=0.045cm]{850}{1400}}
\rput{90}(0,0){\psCoil[doubleline=true,linecolor=black,linewidth=0.07cm]{1260}{1550}}
\rput{90}(0,0){\psCoil[doubleline=true,linecolor=magenta,linewidth=0.045cm]{1220}{1550}}
%---------------------------------------------------------------------------------------------------------------
\psline[linewidth=1.47mm,linecolor=magenta,linearc=0.15,bordercolor=black,border=1.1pt](0.2,1.943)(4.1,1.943)(4.1,1.6)%
\psline[linewidth=1.47mm,linecolor=magenta,linearc=0.15,bordercolor=black,border=1.1pt](4.65,1.6)(4.65,1.943)(5.75,1.943)(5.75,1.6)%
\psline[arrowinset=0,arrowscale=1.2,arrowlength=0.8,linewidth=0.6pt]{->}(5.25,0.5)(5.25,0.1)%
\psline[arrowinset=0,arrowscale=1.2,arrowlength=0.8,linewidth=0.6pt]{->}(6.4,0.5)(6.4,0.1)%
\pscircle[linewidth=0.8pt,fillstyle=solid,fillcolor=blue!20!green!70](8.3,-0.4){0.17}%
\psline[linewidth=1.3mm,linecolor=blue!20!green!70,linearc=0.15,bordercolor=black,border=1.1pt]%
(8.18,-0.4)(5.1,-0.4)(5.1,-0.1)(9.4,-0.1)(9.4,-1.2)%
\psline[linewidth=1.3mm,linecolor=blue!20!green!70,linearc=0.15,bordercolor=black,border=1.1pt]%
(8.423,-0.4)(9.1,-0.4)(9.1,-1.2)%
%-------------------------------------------------------------------------------------------
%--------------------------- Fluss ----------------------------------------------------------
\psplot[linecolor=blue]{8.9}{9.8}{x 1600 mul sin 0.02 mul 1.05 sub}%
\psplot[linecolor=blue]{8.9}{9.8}{x 1600 mul sin 0.02 mul 1.15 sub}%
\psplot[linecolor=blue]{8.9}{9.8}{x 1600 mul sin 0.02 mul 1.25 sub}%
\psplot[linecolor=blue]{8.9}{9.8}{x 1600 mul sin 0.02 mul 1.35 sub}%
\psplot[linecolor=blue]{8.9}{9.8}{x 1600 mul sin 0.02 mul 1.45 sub}%
\psplot[linecolor=blue]{8.9}{9.8}{x 1600 mul sin 0.02 mul 1.55 sub}%
%-------------------------------------------------------------------------------------------
%-------------------------- Turbine ---------------------- ----------------------------------
\psframe[fillstyle=gradient,linecolor=black,gradmidpoint=0,gradangle=0,gradbegin=blue!20!green!70,gradend=green!10]%
(3.9,0.9)(6.6,1.6)%
\psline(4.9,0.9)(4.9,1.6)%
%---------------------------------Achse und Turbinenr�der ----------------------------------------------------------
\psframe[gradientHSB=true,linestyle=none,fillstyle=gradient,linecolor=black,gradmidpoint=0.4,%
gradangle=0,gradbegin=ColorC,gradend=ColorD](3.7,1.2)(8.47,1.3)%  Achse
%-------------------------------------------------------------------------------------------
\pspolygon[gradientHSB=true,linestyle=none,fillstyle=gradient,linecolor=black,gradmidpoint=0.5,%
gradangle=0,gradbegin=ColorC,gradend=ColorD](4,1.35)(4.8,1.5)(4.8,1.0)(4,1.15)%
\pspolygon[fillstyle=vlines, hatchangle=0, hatchsep=2pt](4,1.35)(4.8,1.5)(4.8,1.0)(4,1.15)%
%-------------------------------------------------------------------------------------------
\pspolygon[gradientHSB=true,linestyle=none,fillstyle=gradient,linecolor=black,gradmidpoint=0.5,%
gradangle=0,gradbegin=ColorC,gradend=ColorD](5.0,1.5)(5.7,1.35)(5.70,1.15)(5.0,1.0)%
\pspolygon[fillstyle=vlines, hatchangle=0, hatchsep=2pt](5.0,1.5)(5.7,1.35)(5.70,1.15)(5.0,1.0)%
%-------------------------------------------------------------------------------------------
\pspolygon[gradientHSB=true,linestyle=none,fillstyle=gradient,linecolor=black,gradmidpoint=0.5,%
gradangle=0,gradbegin=ColorC,gradend=ColorD](5.8,1.35)(6.5,1.5)(6.5,1.0)(5.8,1.15)%
\pspolygon[fillstyle=vlines, hatchangle=0, hatchsep=2pt](5.8,1.35)(6.5,1.5)(6.5,1.0)(5.8,1.15)%
%-------------------------------------------------------------------------------------------
\psframe[fillstyle=solid,linecolor=black,fillcolor=blue!20!green!70,dimen=inner,opacity=0.6](3.8,1.05)(3.9,1.45)%
\psframe[fillstyle=solid,linecolor=black,fillcolor=blue!20!green!70,dimen=inner,opacity=0.6](6.6,1.1)(6.7,1.4)%
%-------------------------------------------------------------------------------------------
%--------------------------------- Generator und Erreger ------------------------------------
\psframe[fillstyle=gradient,linecolor=black,gradmidpoint=0,gradangle=0,gradbegin=blue!60!green!70,gradend=green!10,dimen=inner]%
(6.95,1.1)(7,1.4)%
\psframe[fillstyle=gradient,linecolor=black,gradmidpoint=0,gradangle=0,gradbegin=blue!60!green!70,gradend=green!10]%
(7,0.9)(8,1.6)%
\psframe[fillstyle=gradient,linecolor=black,gradmidpoint=0,gradangle=0,gradbegin=yellow!90,gradend=yellow!20,dimen=inner]%
(8,1.05)(8.4,1.45)%
%\psframe[fillstyle=solid,linecolor=black,fillcolor=black](8.4,1.2)(8.47,1.3)%
%\psframe[fillstyle=solid,linecolor=black,fillcolor=black](6.7,1.2)(7,1.3)%
\psline[linecolor=black,linewidth=0.8pt](7.7,0.9)(7.7,0.8)(8.3,0.8)%
\psline[linecolor=black,linewidth=0.8pt](7.6,0.9)(7.6,0.7)(8.3,0.7)%
\psline[linecolor=black,linewidth=0.8pt](7.5,0.9)(7.5,0.6)(8.3,0.6)%
\rput[l](8.38,0.8){\psplot[linecolor=black,linewidth=0.6pt]{-0.01}{0.16}{x 2500 mul sin 0.02 mul}}
\rput[l](8.38,0.7){\psplot[linecolor=black,linewidth=0.6pt]{-0.01}{0.16}{x 2500 mul sin 0.02 mul}}
\rput[l](8.38,0.6){\psplot[linecolor=black,linewidth=0.6pt]{-0.01}{0.16}{x 2500 mul sin 0.02 mul}}
%-------------------------------------------------------------------------------------------
\pspolygon[fillstyle=solid,fillcolor=black](8.2,-0.4)(8.36,-0.31)(8.36,-0.49)%
\psframe[fillstyle=vlines,hatchangle=90,hatchsep=1.5pt,hatchcolor=red,linewidth=0.8pt](2.2,-0.8)(2.8,-0.2)%
\pscircle[linewidth=0.8pt,fillstyle=solid,fillcolor=blue!70](3.2,-1.1){0.25}%
\pspolygon[fillstyle=solid,fillcolor=black](3.05,-1.1)(3.3,-0.96)(3.3,-1.24)%
\psline[linewidth=1.47mm,linecolor=blue!70,linearc=0.15,bordercolor=black,border=1.1pt](0.2,0.035)(2.5,0.035)(2.5,-1.1)(2.99,-1.1)%
\psline[linewidth=1.47mm,linecolor=blue!70,linearc=0.15,bordercolor=black,border=1.1pt](3.4,-1.1)(5.8,-1.1)(! 5.8 -2 15 sin mul 0.2 sub 0.0175 add)%  0,5 pt addieren, denn das ist die halbe Linienst�rke
\psline[arrowinset=0,arrowscale=1.2,arrowlength=0.8,linewidth=0.6pt]{->}(1.6,0.035)(1.2,0.035)%
\psline[arrowinset=0,arrowscale=1.2,arrowlength=0.8,linewidth=0.6pt]{->}(0.2,0.82)(-0.2,0.85)%
\psline[arrowinset=0,arrowscale=1.2,arrowlength=0.8,linewidth=0.6pt]{->}(1.2,1.943)(1.6,1.943)%
\psline[arrowinset=0,arrowscale=1.2,arrowlength=0.8,linewidth=0.6pt]{->}(5.025,1.943)(5.425,1.943)%
\psline[arrowinset=0,arrowscale=1.2,arrowlength=0.8,linewidth=0.6pt]{->}(7.5,-0.4)(7.1,-0.4)%
\psline[arrowinset=0,arrowscale=1.2,arrowlength=0.8,linewidth=0.6pt]{->}(7.5,-0.1)(7.9,-0.1)%
%-----------------------------------------------------------------------------------------
{\tiny
\rput[t](-0.85,3,1){\shortstack[l]{Rauch-\strut\\[-3pt]\ \quad
gas-\strut\\[-3pt]\qquad kanal\strut}}
\rput[c](-1.2,-2){Brenner}
\psline[linewidth=0.6pt](-1.2,-1.9)(-1.2,-1.2)%
\rput[c](0.25,-2){Brennkammer}
\psline[linewidth=0.6pt](0.25,-1.9)(0.25,-1.2)%
\rput[l](2.9,-1.5){Pumpe}%
\rput[lc](3.2,-0.3){\shortstack[l]{Vor-\strut\\[-3pt]
w�rm-\strut\\[-3pt] anlage}}
\rput[c](2.7,2.7){Frischdampf}
\psline[linewidth=0.6pt](2.7,1.95)(2.7,2.55)%
\psline[linewidth=0.6pt](1.77,0.035)(1.77,0.22)%
\rput[b](1.77,.25){\shortstack{Speise-\strut\\[-3pt]wasser\strut}}
\rput[c](4.4,2.7){\shortstack[c]{Hochdruck-\strut\\[-3pt]turbine\strut}}
\psline[linewidth=0.6pt](4.4,1.5)(4.4,2.5)%
\rput[c](6.1,2.7){\shortstack{Niederdruck-\strut\\[-3pt]turbine\strut}}
\psline[linewidth=0.6pt](6.1,1.5)(6.1,2.5)%
\rput[c](7.5,2.7){Generator}
\psline[linewidth=0.6pt](7.5,1.5)(7.5,2.5)%
\rput[c](8.15,2.3){\shortstack{Erreger-\strut\\[-3pt]maschine\strut}}
\psline[linewidth=0.6pt](8.15,1.4)(8.15,2.1)%
\rput[l](3.8,0.1){\shortstack{Konden-\strut\\[-3pt]sator\strut}} \psline[linewidth=0.6pt](4.5,0.1)(5.0,0.1)%
\rput[c](7.7,-0.8){K�hlwasser}
\psline[linewidth=0.6pt](7.7,-0.4)(7.7,-0.6)%
\rput[l](9.1,-1.8){Fluss}%
}
}

\rput[l](-1,-4.5){%
\psset{ArrowFill=true,arrowinset=0,arrowscale=0.7,arrowlength=0.5,framearc=0.05,dimen=outer}%
%\psframe(14,-0.8)(16,0.8)%
\psline[linewidth=0.7cm,linecolor=gray!40]{->}(12,0)(14,0)%
\psline[linewidth=0.2cm,linecolor=gray!40,linearc=0.3]{->}(12,-0.35)(12.5,-0.35)(12.5,-1.0)
\psframe(10,-0.8)(12,0.8)%
\psline[linewidth=0.9cm,linecolor=gray!40]{->}(8,0)(10,0)%
\psline[linewidth=0.2cm,linecolor=gray!40,linearc=0.3]{->}(8,-0.45)(8.5,-0.45)(8.5,-1.1)
\psframe(6,-0.8)(8,0.8)%
\psline[linewidth=1.1cm,linecolor=gray!40]{->}(4,0)(6,0)%
\psline[linewidth=0.2cm,linecolor=gray!40,linearc=0.3]{->}(4,-0.55)(4.5,-0.55)(4.5,-1.2)
\psframe(2,-0.8)(4,0.8)%
\psline[linewidth=1.3cm,linecolor=gray!40]{->}(0,0)(2,0)%
\psline[linewidth=0.2cm,linecolor=gray!40,linearc=0.3]{->}(0,-0.65)(0.5,-0.65)(0.5,-1.3)
\psframe(-2,-0.8)(0,0.8)%
\rput[c](3,0.3){\psframebox[linecolor=white]{Rohre}}%
\rput[c](3,-0.3){\psframebox[linecolor=white]{Kessel}}%
\rput[l](4.1,0.2){\footnotesize potentielle}%
\rput[l](4.1,-0.3){\footnotesize Energie}%
\textcolor{red}{%
\rput[l](0.1,0.2){\footnotesize thermische}%
\rput[l](0.1,-0.3){\footnotesize Energie}%
\rput[l](8.1,0.2){\footnotesize kinetische}%
\rput[l](8.1,-0.3){\footnotesize Energie}%
\rput[l](12.1,0){\footnotesize elektr. Energie}%
\rput[c](-1,0){\psframebox[linecolor=white]{Brenner}}%
\rput[c](7,0){\psframebox[linecolor=white]{Turbine}}%
\rput[c](11,0){Generator}%
\rput[l](12.7,-0.85){\footnotesize WE}%
\rput[l](8.7,-0.95){\footnotesize WE}%
\rput[l](4.7,-1.05){\footnotesize WE}%
\rput[l](0.7,-1.15){\footnotesize WE}%
}}%


\end{document} 