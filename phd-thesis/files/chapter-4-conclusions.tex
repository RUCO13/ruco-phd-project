\chapter{Conclusions and Future Work }
\label{chap:Intro}
\textit{In this chapter raise the principal conclusion of this work and present the future experiments on these structures in the wake of obtained results.}
\vfill
\minitoc
\newpage
\allowdisplaybreaks

\lettrine[lines=3, lraise=.1, nindent=0mm, slope=0mm]{\textbf{T}}{he} aim of this work finally results in important publications, the \gls{CQWs} continue to be an excellent platform to study optical and quantum-mechanical properties, this work presents an important result and enhance the importance of study these structures. The principal idea to exposes our results was planned to simplify but specify the physical basis, starting from explain a single QW and structural properties then raise the relevance of symmetry context to understand the physical behavior of electrons. In section exposes the symmetry importance in this work and their fundamental role in emergence of the \gls{oa}. Then focus on symmetry reduction (symmetry breaking)  which is the causes of appearance  of interesting physical properties, in our case optical properties. 

The perturbative model purposed to understand of the \gls{oa} is simple and useful, as this depend on grade of asymmetry in the CQWs system, the more asymmetry increases the RAS signal. Also, this model is support by the numerical calculations with a good approximation. When it had obtained the firsts results, we purposed to developed  codes to generate numerical results and although this represented a new area to explore, we decided to dedicated enough time to get a new tool to support our experimental work. Already inside in numerical solutions' area, we realized of complexity of  generate reliable results, overall to existence of numerous proposals to get it, then we decided implemented the simplicity context. Our numerical results are simple but  reliable in accordance with the numerical results, by this reason we create a \href{https://github.com/lflmgroup}{GitHub} repository\cite{lflmgroup} with aim to developed new codes and numerical models which in future work can will be implemented. In the experimental part, the results are the proof of the arduously work that it was inverted, along of the project it was realized experiments to study and understand the physics which involves the \gls{CQWs} systems, although this has been study for several years ago, our contribution it's novel. \\*
The quantum confinement is the key of these structures, if add the symmetry breaking by coupled wells width asymmetry exhibits wonderful physics, from spin dynamics to excitonic effects. With respect to spin dynamics into CQWs structures, it was realized experiments of circularly polarized PL over the samples shown in this work, reveals that the spin relaxation time $\tau_{s}$. It has been demonstrates that the degree of circular polarization is directly related to the asymmetry of the \gls{CQWs}\cite{bravo2022photoluminiscence}. \\*
The excitonic properties shown in the \Cref{subsubsec:chapter-3-PR-exciton-effects} are really interesting, the \gls{PR} experiments performed as a laser power function reveal a non-common transition in these experiments, this transition associated with a trion, commonly occurs in structures under external disturbance as an electric field applied, in our case, we not only detect the trion transition, but also it modulated with a light source.  This means this can be applied as a laser transistor. These  PR results are very relevant, in fact as a future work we planned to publish them. 


Finally, in the RAS experiments  it's clearly the wonderful physics which exhibits ACQWs structures, the results obtained are the principle of an experiments' series  which we're thinking to carried out. Without intention to being  repetitive since the results proofs our hard work, we purpose to take further the RAS technique to explore with more detail the ACQWs structures. The first upgrade of RAS experiments, it's do it spin sensitive, this means, spin resolve RAS experiments. The principal idea is enhancing the RAS setup to measure spin response. Our proposal is to carry out the experiments just changing the modulated PEM polarization.  